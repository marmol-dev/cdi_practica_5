\begin{quote}
Martín Molina Álvarez \end{quote}


\begin{quote}
Víctor Malvárez Filgueira \end{quote}


\subsection*{Modo de empleo}


\begin{DoxyEnumerate}
\item Compilar los {\ttfamily .java}.
\item Ir al directorio donde están los {\ttfamily .class}.
\item Abrir el \hyperlink{classServidor}{Servidor} con los argumentos necesarios\+:
\begin{DoxyItemize}
\item puerto
\item divisiones
\item x\+Centro
\item y\+Centro
\item tamaño
\item iteraciones
\item archivo
\end{DoxyItemize}
\item Ejemplo\+: {\ttfamily java \hyperlink{classServidor}{Servidor} 3000 16 512 512 1024 512 imagen.\+pgm}
\item Abrir los clientes\+: {\ttfamily java \hyperlink{classCliente}{Cliente} ...} Los argumentos son\+:
\begin{DoxyItemize}
\item hostname\+: la dirección del servidor
\item puerto
\item n\+Clientes\+: Opcional.
\end{DoxyItemize}
\item Ejemplo\+: {\ttfamily java \hyperlink{classCliente}{Cliente} localhost 3000 16}
\end{DoxyEnumerate}

\subsection*{Arquitectura}

\subsubsection*{\hyperlink{classServidor}{Servidor}}

Clase principal del servidor que se encarga de abrir las conexiones con los clientes y de gestionar el proceso de organización y control de las colas.

\subsubsection*{\hyperlink{classServidorThread}{Servidor\+Thread}}

Clase cuyas instancias son creadas cada vez que se conecta un nueco cliente. Gestiona y controla una conexión individual con un cliente.

\subsubsection*{\hyperlink{classCliente}{Cliente}}

Clase cuyas instancias son creadas una vez que se ha abierto el servidor. Hacen las llamadas al algoritmo de \hyperlink{classMandelbrot}{Mandelbrot}.

\subsubsection*{\hyperlink{classTrabajo}{Trabajo}}

Clase que contiene los atributos necesarios para procesar una región de la imagen del \hyperlink{classMandelbrot}{Mandelbrot}.

\subsubsection*{Acción}

Clase vehicular para comunicarse entre \hyperlink{classCliente}{Cliente} y \hyperlink{classServidor}{Servidor}.

\subsubsection*{\hyperlink{classMandelbrot}{Mandelbrot}}

Clase cuyo método implementa el algoritmo de \hyperlink{classMandelbrot}{Mandelbrot}.

\subsubsection*{\hyperlink{classPGM}{P\+GM}}

Clase que se encarga de la creación de imágenes con el mismo formato que el nombre de la clase y escribe matrices de colores en formato R\+GB. 